% This is "aamas2013 .tex" September 2012 
% This file should be compiled with "aamas2013 .cls" 
% This example file demonstrates the use of the 'aamas2013 .cls'
% LaTeX2e document class file. It is for those submitting
% articles to AAMAS 2013  conference. This file is based on
% the sig-alternate.tex example file.
% The 'sig-alternate.cls' file of ACM will produce a similar-looking,
% albeit, 'tighter' paper resulting in, invariably, fewer pages.
% than the original style ACM style.
%
% ----------------------------------------------------------------------------------------------------------------
% This .tex file (and associated .cls ) produces:
%       1) The Permission Statement
%       2) The Conference (location) Info information
%       3) The Copyright Line with AAMAS data
%       4) NO page numbers
%
% as against the acm_proc_article-sp.cls file which
% DOES NOT produce 1) thru' 3) above.
%
% Using 'aamas2013 .cls' you don't have control
% from within the source .tex file, over both the CopyrightYear
% (defaulted to 200X) and the IFAAMAS Copyright Data
% (defaulted to X-XXXXX-XX-X/XX/XX).
% These information will be overwritten by fixed AAMAS 2013  information
% in the style files - it is NOT as you are used with ACM style files.
%
% ---------------------------------------------------------------------------------------------------------------
% This .tex source is an example which *does* use
% the .bib file (from which the .bbl file % is produced).
% REMEMBER HOWEVER: After having produced the .bbl file,
% and prior to final submission, you *NEED* to 'insert'
% your .bbl file into your source .tex file so as to provide
% ONE 'self-contained' source file.
%

\newcommand{\bE}{\mathbb{E}}
\newcommand{\cI}{\mathcal{I}}
\newcommand{\cC}{\mathcal{C}}
\newcommand{\defword}[1]{\textbf{\boldmath{#1}}}

% This is the document class for full camera ready papers and extended abstracts repsectively 

\documentclass{aamas2013}

% if you are using PDF LaTex and you cannot find a way for producing
% letter, the following explicit settings may help
 
\pdfpagewidth=8.5truein
\pdfpageheight=11truein

\begin{document}

% In the original styles from ACM, you would have needed to
% add meta-info here. This is not necessary for AAMAS 2013  as
% the complete copyright information is generated by the cls-files.

\title{Further Developments in Extensive-Form Replicator Dynamics using Sequence Form Representations}

% AUTHORS


% For initial submission, do not give author names, but the
% tracking number, instead, as the review process is blind.

% You need the command \numberofauthors to handle the 'placement
% and alignment' of the authors beneath the title.
%
% For aesthetic reasons, we recommend 'three authors at a time'
% i.e. three 'name/affiliation blocks' be placed beneath the title.
%
% NOTE: You are NOT restricted in how many 'rows' of
% "name/affiliations" may appear. We just ask that you restrict
% the number of 'columns' to three.
%
% Because of the available 'opening page real-estate'
% we ask you to refrain from putting more than six authors
% (two rows with three columns) beneath the article title.
% More than six makes the first-page appear very cluttered indeed.
%
% Use the \alignauthor commands to handle the names
% and affiliations for an 'aesthetic maximum' of six authors.
% Add names, affiliations, addresses for
% the seventh etc. author(s) as the argument for the
% \additionalauthors command.
% These 'additional authors' will be output/set for you
% without further effort on your part as the last section in
% the body of your article BEFORE References or any Appendices.

%\numberofauthors{8} %  in this sample file, there are a *total*
% of EIGHT authors. SIX appear on the 'first-page' (for formatting
% reasons) and the remaining two appear in the \additionalauthors section.
%

\numberofauthors{1}

\author{
% You can go ahead and credit any number of authors here,
% e.g. one 'row of three' or two rows (consisting of one row of three
% and a second row of one, two or three).
%
% The command \alignauthor (no curly braces needed) should
% precede each author name, affiliation/snail-mail address and
% e-mail address. Additionally, tag each line of
% affiliation/address with \affaddr, and tag the
% e-mail address with \email.
% 1st. author
\alignauthor
Paper  XXX
%Ben Trovato\titlenote{Dr.~Trovato insisted his name be first.}\\
%       \affaddr{Institute for Clarity in Documentation}\\
%       \affaddr{1932 Wallamaloo Lane}\\
%       \affaddr{Wallamaloo, New Zealand}\\
%       \email{trovato@corporation.com}
% 2nd. author
%\alignauthor
%G.K.M. Tobin\titlenote{The secretary disavows any knowledge of this author's actions.}\\
%       \affaddr{Institute for Clarity in Documentation}\\
%       \affaddr{P.O. Box 1212}\\
%       \affaddr{Dublin, Ohio 43017-6221}\\
%       \email{webmaster@marysville-ohio.com}
% 3rd. author
%\alignauthor Lars Th{\o}rv{\"a}ld\titlenote{This author is the one who did all the really hard work.}\\
%       \affaddr{The Th{\o}rv{\"a}ld Group}\\
%       \affaddr{1 Th{\o}rv{\"a}ld Circle}\\
%       \affaddr{Hekla, Iceland}\\
%       \email{larst@affiliation.org}
}

%\and  % use '\and' if you need 'another row' of author names

% 4th. author
%\alignauthor Lawrence P. Leipuner\\
%       \affaddr{Brookhaven Laboratories}\\
%       \affaddr{Brookhaven National Lab}\\
%       \affaddr{P.O. Box 5000}\\
%       \email{lleipuner@researchlabs.org}

% 5th. author
%\alignauthor Sean Fogarty\\
%       \affaddr{NASA Ames Research Center}\\
%       \affaddr{Moffett Field}\\
%       \affaddr{California 94035}\\
%       \email{fogartys@amesres.org}

% 6th. author
%\alignauthor Charles Palmer\\
%       \affaddr{Palmer Research Laboratories}\\
%      \affaddr{8600 Datapoint Drive}\\
%       \affaddr{San Antonio, Texas 78229}\\
%       \email{cpalmer@prl.com}

%\and

%% 7th. author
%\alignauthor Lawrence P. Leipuner\\
%       \affaddr{Brookhaven Laboratories}\\
%       \affaddr{Brookhaven National Lab}\\
%       \affaddr{P.O. Box 5000}\\
%       \email{lleipuner@researchlabs.org}

%% 8th. author
%\alignauthor Sean Fogarty\\
%       \affaddr{NASA Ames Research Center}\\
%       \affaddr{Moffett Field}\\
%       \affaddr{California 94035}\\
%       \email{fogartys@amesres.org}

%% 9th. author
%\alignauthor Charles Palmer\\
%       \affaddr{Palmer Research Laboratories}\\
%       \affaddr{8600 Datapoint Drive}\\
%       \affaddr{San Antonio, Texas 78229}\\
%       \email{cpalmer@prl.com}

%}

%% There's nothing stopping you putting the seventh, eighth, etc.
%% author on the opening page (as the 'third row') but we ask,
%% for aesthetic reasons that you place these 'additional authors'
%% in the \additional authors block, viz.
%\additionalauthors{Additional authors: John Smith (The Th{\o}rv{\"a}ld Group,
%email: {\texttt{jsmith@affiliation.org}}) and Julius P.~Kumquat
%(The Kumquat Consortium, email: {\texttt{jpkumquat@consortium.net}}).}
%\date{30 July 1999}
%% Just remember to make sure that the TOTAL number of authors
%% is the number that will appear on the first page PLUS the
%% number that will appear in the \additionalauthors section.

\maketitle

\begin{abstract}
Evolutionary game theory has been used to model interactions in populations of rational agents. 
However, research in complex interactions, as modeled by general extensive-form games, has been researched 
significantly less so than its normal-form ``one-shot'' counterpart. 
Recently, replicator dynamics have been adapted to extensive-form games represented in sequence form, leading to 
a large reduction in computational resources. 
However, the properties of systems following these dynamics remain unknown.
In this paper, we provide the first evidence of convergence to equilibrium strategies. In additon, we generalize the
replictor dynamics to more than two players, and show that they can find equilibrium points in a 3-player game as well. 
Finally we analyze the stability of the solutions and derive new dynamics that can be used to approach equilibrium 
refinements, such as sequential equilibria. 
\end{abstract}

% Note that the category section should be completed after reference to the ACM Computing Classification Scheme available at
% http://www.acm.org/about/class/1998/.

\category{I.2.11}{Distributed Artificial Intelligence}{Multiagent systems}

%A category including the fourth, optional field follows...
%\category{D.2.8}{Software Engineering}{Metrics}[complexity measures, performance measures]

%General terms should be selected from the following 16 terms: Algorithms, Management, Measurement, Documentation, Performance, Design, Economics, Reliability, Experimentation, Security, Human Factors, Standardization, Languages, Theory, Legal Aspects, Verification.

\terms{Algorithms, Economics, Theory}

%Keywords are your own choice of terms you would like the paper to be indexed by.

\keywords{Replicator dynamics, game theory, extensive form games, multiplayer, Nash equilibrium, sequence form}

% PAPER NOTES: 

% Work that should be cited: 
%  - Gatti's paper
%  - Text book: EGT in Extensive Form Games (R. Cressman 2003)
%  - Klos 2010 Evolutionary Dynamics of Regret Minimization (as future work), saved in papers/Klos10Evo... 
%    - maybe a tie to CFR 
%  - Frank's recent paper Evolutionary Stochastic Games (J. Flesh), saved in papers/Flesch13Evolut...
%  - ?? (Not sure.. doesn't seem to be about extensive form) Extended replicator dynamics ... papers/tuyls07extended
%  - 3P Kuhn 
%  - Evolutionary analysis of poker play (see Karl's web site), in particular "An Evolutionary Game-Theoretic Analysis of Poker Strategies"
%    in Entertainment Computing, which used data from real Poker games (saved as papers/Ponsen09An)
%  - "What evolutionary game theory tells us about multiagent learning" AIJ article
%    - possible James Wright's behavioral game theory paper
%  - text book theory of learning in games, Fudenberg & Levin

\section{Introduction}

%Main points:
%\begin{itemize}
%\item Evolutionary game theory and extensive form games~\cite{Cressman03}
%\item Recent developments make this efficient~\cite{Gatti13Efficient} 
%\item However, no empricial evidence?!
%\item Extensions to equilibrium refinements~\cite{Miltersen06Computing} and multiplayer
%\item Empirical evaulation in complex games
%\end{itemize}

Evolutionary game theory~\cite{MaynardSmith82,Gintis09} has been used to model and explain complex interactions in multiagent 
systems such as population dynamics~\cite{HS98}, animalistic behavior in nature~\cite{MSP73}, and 
multiagent learning~\cite{Tuyls07What,Tuyls03Selection}. The most popular and widely-studied population dynamic is the
so-called {\it replicator dynamic}~\cite{TJ78}. Generally, replicator dynamics quantify 
increases in the proportion of individuals in a population based on their relative fitness levels as determined through 
payoffs of competitions. Agents with higher fitness replicate more than agents with lower fitness and the resulting 
process leads to an evolutionary game that models population change under these dynamics. 

Multiagent systems that evolve under replicator dynamics have desirable properties and connections to classical game theory. 
For example, fixed points of populations under the replicator dynamics correspond to a Nash equilibrium of the underlying normal 
form game and stability of the system can be analyzed using theory of dynamical systems~\cite{Gintis09}. In addition, average payoff 
of a population increases~\cite{HS98} and dominated strategies do not survive~\cite{Gintis09}. 

In the classic setup, the underlying stage game is a symmetric normal-form game. However, this limits the interaction
among the agents and payoffs are determined from single decisions made by each agent. In general, the games played among agents 
can be more complex, such as multi-step games such as extensive-form games. Evolutionary dynamics have been extended to multi-stage models such as extensive-form games~\cite{Cressman03} and stochastic games~\cite{Hennes09State,Flesch13Evolutionary}, 
however the focus has been mainly on subgame-decomposable formalisms such as perfect information games and
simultenous move games. In the fully general setting of imperfect information, the extensive-form game cannot be easily 
decomposed into smaller subgames, making the problem significantly more challenging.

Recently, efficient replicator dynamics have been proposed for the general case of extensive-form stage games with 
imperfect information~\cite{Gatti13Efficient}, based on sequence-form representations~\cite{SequenceFormLPs}. 
In their paper, the authors present discrete and continuous time replicator dynamics
for extensive-form imperfect information games that can be represented with much less computational requirements than their equivalent
normal forms. However, the authors do not present experimental results nor do they consider the case beyond two players. 
In this paper, we present the first experimental evidence confirming the convergence of the sequence-form replicator dynamics  
to a Nash equilibrium in two-player Kuhn poker. We then extend these replicator dynamics to $n$-player games and show convergence 
properties in three-player Kuhn poker. Finally, we provide conditions under which convergence to an equilibrium can be expected by relating the 
dynamics to no-regret learning algorithms used to compute approximate equilibrium strategies in self-play. 


%, based on sequence-form representations~\cite{SequenceFormLPs}. 

\section{Background}

%Evolutionary game theory. 
%Replicator dynamics.
%Extensive-form games. 
%Sequence form representation.
%Kuhn Poker. 

In this section we define the relevant game-theoretic terminology that forms the basis
of our analysis. The notation used here is based on~\cite{OsbRub94}. 

A normal-form game is a one where each agent $i \in N = \{ 1, \ldots, n \}$ 
chooses a single \defword{pure strategy} $s_i \in S_i$ simultaneously, where $N$ and $S_i$ are assumed to 
be finite. The payoff to each agent when strategy profile $s$ is employed is denoted $u_i(s)$, where $s \in S$, and 
$S = S_1 \times S_2 \times \ldots \times S_n$ is the space of pure strategy profiles. 
When $|N| = 2$ these are called {\it matrix games} since the payoffs $u_i(s)$ can be represented as a matrix. 
A normal-form game is \defword{symmetric} if and only if $S_i = S_j$ for $i \not= j$. Let $\Sigma_i$ denote the set of 
all probability distributions over $S_i$ and $\sigma_i \in \Sigma_i$ a \defword{mixed strategy} for player $i$. 
Then, the expected payoff for player $i$ is denoted $u_i(\sigma) = \bE_{s \sim \sigma}[u(s)] = \sum_{s \in S} \sigma(s) u(s)$
where $\sigma$ is a mixed strategy profile and $\sigma(s)$ is the probability of playing $s$. 

An extensive-form game models sequential decision making. In this case, there are \defword{histories} $h \in H$ which represent
sequences of \defword{actions}. A history $z \in Z$, where $Z \subseteq H$, is called a \defword{terminal history} marking the end of the 
game. At each terminal history $z$ there is a payoff $u_i(z)$ to player $i$. At each nonterminal history $h$, there is a single 
current player to act, $P: H \backslash Z \rightarrow N \cup \{ c \}$ where $c$ is a special player called \defword{chance}
(sometimes also called nature) that plays with a fixed stochastic strategy. For example, chance is used to represent die rolls 
and card draws. The game starts in the empty history, and 
at each step, given the history $h$, the current player chooses an action $a \in A(h)$ leading to successor history $h' = ha$;
in this case we call $h$ a \defword{prefix} of $h'$ and denote this relationship by $h \sqsubset h$.

Denote $\cI$ the global information partition with disjoint parts $\cI_i$ for each player $i$. Intuitively, an \defword{information set} 
$I \in \cI_i$ that belongs to player $i$ represents a state of the game with respect to what player $i$ knows. 
Formally, $I$ is a set of histories that a player cannot distinguish (due information hidden from that player). For all 
$h,h' \in I$, $A(h) = A(h')$ and $P(h) = P(h')$; hence sometimes we use $A(I)$ and $P(I)$. 

We also define the set of (information set, action) pairs for one player to be
$\cC_i = \{ (I,a) \mid I \in \cI_i, a \in A(I) \}$.
%and $I(h)$ as the information set containing the nonterminal history $h$. 
For a history $h \in H$, define
$X_i(h) = ((I,a), (I', a'), (I'', a''), \cdots)$ to be the sequence of player $i$'s (information set,
action) pairs that were encountered and taken to reach $h$ in the same order as they are encountered
and taken in $h$. An extensive-form game has \defword{perfect recall} if
$\forall i \in N, \forall I \in \cI_i : h, h' \in I \Rightarrow X_i(h) = X_i(h')$. Intuitively,
this means that player $i$ does not forget any information that they discovered during their play
up to $h$.

%equilibrium definitions
When $|N| = 2$ and $u_1(z) = -u_2(z)$ for all $z \in Z$, 
then the game is a two-player zero-sum game. Two-player zero-sum games form an important subclass of extensive-form games due to
worst-case guarantees 

\section{Observed Convergence Analysis}

\section{New Extensions}

\subsection{Equilibrium Refinements}

\subsection{Multiplayer Games}

\section{Empirical Evaluation}

\section{Conclusion}


\bibliographystyle{plain}
\bibliography{sfegt}

\end{document}

